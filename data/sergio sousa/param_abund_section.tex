%
\documentclass[]{aa} % for the letters 
%
%\documentclass[structabstract]{aa}  
%\documentclass[traditabstract]{aa} % for the abstract without structuration 
%\usepackage{subfigure}                                   % (traditional abstract) 
\usepackage{epsfig}
\usepackage{wasysym}
\usepackage{graphicx}
\usepackage{natbib}
\usepackage{booktabs}
%\usepackage{amssymb, amsmath}
\usepackage{lscape}
%\usepackage{multirow}

\bibliographystyle{aa}

\usepackage{url}
\usepackage[breaklinks]{hyperref}

%\usepackage{amsmath}
\usepackage{color}
\providecommand{\tabularnewline}{\\}
%%%%%%%%%%%%%%%%%%%%%%%%%%%%%%%%%%%%%%%%
\usepackage{txfonts}
%%%%%%%%%%%%%%%%%%%%%%%%%%%%%%%%%%%%%%%%
%

  
\begin{document}

  \title{Stellar parameters}

  \author{         }

  \institute{}


  \date{Received date / Accepted date }
% \abstract{}{}{}{}{} 
% 5 {} token are mandatory
 
%----------------------------------------------------------------------------------------
%       Abstract
%----------------------------------------------------------------------------------------
  \abstract
  {}
  {.}
  {.}
  {.}
  {.}
  
  \keywords{}


  
\maketitle

%
%________________________________________________________________________________________________________



\section{Stellar parameters and chemical abundances}            \label{sec:parameters}

The stellar athmospheric parameters ($T_{\mathrm{eff}}$, $\log g$, microturbulence, [Fe/H]) were derived using the ARES+MOOG methodology described in \citet[][]{Sousa-21, Sousa-14, Santos-13}. We use the latest version of ARES \footnote{The last version, ARES v2, can be downloaded at https://github.com/sousasag/ARES} \citep{Sousa-07, Sousa-15} to measure the equivalent widths (EW) of selectd iron lines on the combined spectrum of TOI-1199 and TOI-1273. The list of iron lines is the same as the one presented in \citet[][]{Sousa-08}. A minimization process is used to find the ionization and excitation equilibrium and converge to the best set of spectroscopic parameters. This process makes use of a grid of Kurucz model atmospheres \citep{Kurucz-93} and the radiative transfer code MOOG \citep{Sneden-73}. We also derived a more accurate trigonometric surface gravity using recent GAIA data following the same procedure as described in \citet[][]{Sousa-21}.


\bibliography{refbib}




\end{document}

